\documentclass[a4paper,12pt]{article}
\usepackage[utf8]{inputenc}
\usepackage[spanish]{babel}
\usepackage{graphicx}
\graphicspath{{./figuras/}}
\usepackage[a4paper, left=2cm, right=2cm, top=2.5cm, bottom=2.5cm]{geometry}
\usepackage{xcolor} % Añadido para usar colores

% Definir un color para los títulos (puedes ajustar el color como quieras)
\definecolor{azulUPP}{RGB}{0,70,128} % Un azul corporativo o tecnológico

% Redefinir el formato de las secciones y subsecciones para incluir color
\usepackage{titlesec}
\titleformat{\section}{\Large\bfseries\color{azulUPP}}{}{0em}{\MakeUppercase}[]
\titleformat{\subsection}{\large\bfseries\color{azulUPP}}{}{0em}{}[]
\titleformat{\subsubsection}{\bfseries\color{azulUPP}}{}{0em}{}[]

\begin{document}
	% Portada
	\begin{titlepage}
		\centering
		% Logos institucionales y nombre de la universidad
		\includegraphics[width=0.25\textwidth]{logoUpp} \hspace{7cm} 
		\includegraphics[width=0.25\textwidth]{logoSftw}
		\vspace{2cm}
		
		\Huge \textbf{Universidad Politécnica de Pachuca}\\
		\vspace{0.5cm}
		\Large Ingeniería en Software
		\vspace{1cm}
		
		% Titulo
		\Huge \textbf{Conceptos Teóricos - Proyecto de Tercer Parcial.}
		\vspace{1cm}
		
		% Información completa
		\Large \textbf{Materia:} Arquitectura de computadoras \\
		\Large \textbf{Docente:} Víctor Hainy\\
		\vspace{1cm}
		\Large \textbf{Alumno:} \\Peña Serrano José Abraham\\
		\vspace{1cm}
		\Large \textbf{Matricula:} \\2331123273\\
		\vspace{1cm}
		\Large \textbf{Grupo:} SFTW\_06\_01\\
		\vspace{1cm}
		\Large \textbf{Cuatrimestre:} Mayo - Agosto 2025
		
	\end{titlepage}
	
	\tableofcontents
	
	\newpage
	\section{Objetivo de aprendizaje}
	Describir el uso y funcionamiento de los microcontroladores, actuadores y sistemas embebido en sistemas reales aplicados.
	
	\section{Conceptos Teóricos}
	
	\subsection{Microcontroladores}
	\subsubsection{Definición}
	Un \textbf{microcontrolador} es un circuito integrado que incluye en un solo chip los elementos esenciales de un sistema informático: una unidad central de procesamiento (CPU), memoria (RAM y ROM/Flash) y periféricos de entrada/salida (E/S). A diferencia de los microprocesadores convencionales, los microcontroladores están diseñados para controlar dispositivos específicos o sistemas embebidos, realizando tareas concretas de manera automática y eficiente. Se encuentran en aplicaciones tan variadas como electrodomésticos, automóviles, sistemas de automatización industrial y dispositivos médicos.
	
	El microcontrolador es el “cerebro” de sistemas electrónicos autónomos, permitiendo la interacción con el entorno y la automatización de tareas específicas sin requerir la intervención continua de un usuario.
	
	\subsubsection{Tipos}
	Existen varios criterios para clasificar los microcontroladores, siendo los más comunes:
	
	\begin{itemize}
		\item \textbf{Por arquitectura:}
		\begin{itemize}
			\item \textit{Basados en arquitectura Harvard:} Separan físicamente la memoria de instrucciones y de datos (por ejemplo, la familia PIC).
			\item \textit{Basados en arquitectura Von Neumann:} Comparten el mismo bus para instrucciones y datos (como algunos modelos de 8051).
		\end{itemize}
		\item \textbf{Por tamaño de palabra (bits):}
		\begin{itemize}
			\item Microcontroladores de 8 bits: Los más extendidos en aplicaciones sencillas y de bajo costo, como el 8051 o los PIC de gama baja.
			\item Microcontroladores de 16 bits: Ofrecen mayor capacidad de procesamiento para sistemas más complejos.
			\item Microcontroladores de 32 bits: Destinados a aplicaciones que requieren alto rendimiento, como ARM Cortex-M usados en IoT y entornos industriales.
		\end{itemize}
		\item \textbf{Por fabricante/familia:}
		\begin{itemize}
			\item 8051: Amplia trayectoria en aplicaciones industriales y educativas.
			\item PIC: Desarrollados por Microchip, populares en electrónica y automatización.
			\item AVR: Utilizados en plataformas Arduino, ideales para prototipado educativo.
			\item ARM: Reconocidos por eficiencia y potencia, usados en aplicaciones comerciales.
		\end{itemize}
	\end{itemize}
	
	\subsubsection{Componentes y Características}
	Los microcontroladores integran los siguientes \textbf{componentes principales}:
	
	\begin{itemize}
		\item \textbf{CPU (Unidad Central de Procesamiento):} La parte que ejecuta instrucciones y procesa datos.
		\item \textbf{Memoria RAM:} Almacenamiento temporal para datos en ejecución.
		\item \textbf{Memoria ROM/Flash:} Almacena el programa y datos permanentes; puede ser reprogramable.
		\item \textbf{Periféricos de Entrada/Salida (E/S):} Puertos digitales, ADC, PWM, interfaces de comunicación (USART, SPI, I2C, CAN, USB, etc.).
		\item \textbf{Temporizadores y contadores:} Para controlar eventos temporales.
		\item \textbf{Convertidores Analógico-Digital (ADC) y Digital-Analógico (DAC):} Para manejo de señales analógicas.
		\item \textbf{Módulos de comunicación:} UART, SPI, I2C, CAN, que permiten interacción con otros dispositivos.
		\item \textbf{Reloj interno o externo:} Para sincronización y control del rendimiento.
	\end{itemize}
	
	Entre sus \textbf{características} destacan:
	
	\begin{itemize}
		\item Bajo consumo energético y modos de ahorro de energía.
		\item Tamaño compacto y bajo costo.
		\item Facilidad de reprogramación y vasto soporte en herramientas de desarrollo.
		\item Resistencia a ambientes adversos con encapsulados específicos.
	\end{itemize}
	
	\subsubsection{Proceso de Selección}
	
	El microcontrolador ESP32-WROOM-32 cumple con los criterios de selección para HYDROWARE, los cuales son:
	
	\begin{itemize}
		\item \textbf{Tipo de aplicación:} El ESP32 es ideal para sistemas IoT (Internet de las Cosas) de monitoreo. Su capacidad para manejar múltiples sensores y la comunicación inalámbrica se adapta perfectamente a la complejidad del sistema HYDROWARE.
		\item \textbf{Cantidad y tipo de entradas/salidas necesarias:} Cuenta con suficientes pines GPIO (General-Purpose Input/Output) para conectar el sensor de temperatura DS18B20, el sensor de pH y el display LCD 2x16. Además, maneja entradas y salidas digitales y analógicas necesarias.
		\item \textbf{Memoria requerida:} El ESP32 dispone de memoria SRAM y Flash que son suficientes para el código del sistema, el registro de datos y la gestión de la aplicación móvil.
		\item \textbf{Velocidad y capacidad de procesamiento:} Su procesador de doble núcleo (Dual-Core) ofrece la capacidad de procesamiento necesaria para leer datos de los sensores, procesar la información y gestionar la conectividad inalámbrica de manera simultánea y eficiente.
		\item \textbf{Interfaces de comunicación:} La integración de Wi-Fi y Bluetooth es crucial para el proyecto. El Wi-Fi permite la comunicación con la aplicación móvil y la visualización de datos en tiempo real, mientras que el Bluetooth podría ser usado para la configuración inicial o comunicación a corta distancia.
		\item \textbf{Consumo de energía:} El ESP32 ofrece modos de bajo consumo (Deep Sleep) que son beneficiosos para optimizar la duración de la batería.
		\item \textbf{Disponibilidad de soporte técnico:} El ESP32 tiene una vasta comunidad de desarrolladores, documentación detallada, librerías y herramientas de programación (IDE de Arduino, PlatformIO), lo que facilita su desarrollo y depuración.
		\item \textbf{Costo y disponibilidad:} Es un microcontrolador de bajo costo y fácil de conseguir en el mercado, lo que lo hace accesible para proyectos como HYDROWARE.
	\end{itemize}
	
	El ESP32-WROOM-32 se alinea con todos los factores clave del proceso de selección, garantizando la eficiencia, funcionalidad y viabilidad del sistema de monitoreo automatizado HYDROWARE.
	\vspace{0.5cm}
	%\noindent \textit{Referencias:} \\
	%Raúl A. Aquino Santos y Marien N. Rivera Gutiérrez, \textit{Microcontroladores. Fundamentos y aplicaciones}. \\
	%Microchip Technology, \textit{What is a Microcontroller?} \\
	%Texas Instruments, \textit{Selecting a Microcontroller}. 
	

	
	\subsection{Instrucciones de microcontroladores}
	\subsubsection{Sintaxis de Instrucciones}
	\begin{itemize}
		\item Describir la sintaxis de instrucciones de microcontroladores.
	\end{itemize}
	\subsubsection{Estructura de las Instrucciones}
	\begin{itemize}
		\item Explicar la estructura de las instrucciones de microcontroladores.
	\end{itemize}
	\subsubsection{Características y Tipos de Instrucciones}
	\begin{itemize}
		\item Describir las características y tipos de instrucciones de microcontroladores.
	\end{itemize}
	\subsubsection{Proceso de Codificación}
	\begin{itemize}
		\item Explicar el proceso de codificación de instrucciones de microcontroladores.
	\end{itemize}
	
	\subsection{Programación de microcontroladores}
	\subsubsection{Interfaces de microcontroladores}
	\begin{itemize}
		\item Reconocer los conceptos de señal analógica y digital.
		\item Reconocer las características de los sistemas digitales y analógicos.
		\item Definir el concepto de interfaz de microcontroladores.
		\item Describir las características y tipos de interfaces de microcontroladores.
		\item Definir el concepto de periférico de entrada y salida vinculado a microcontroladores.
		\item Describir las características y tipos de periféricos de entrada y salida de microcontroladores. Los especificos del proyecto.
		\item Explicar el proceso de conexión de interfaces de entrada y salida de microcontroladores con periféricos.
		\item Explicar el proceso de programación de microcontroladores.
	\end{itemize}
	
	\subsection{Sistemas embebidos}
	\subsubsection{Definiciones}
	\begin{itemize}
		\item Definir los conceptos de sistema embebido, sensor y actuador.
		
		\subsubsection{Características y Tipos}
		\begin{itemize}
			\item Explicar las características y tipos de sistemas embebidos, sensores y actuadores... Diagrama de bloques elemento fisico actuador, etc. Todo es el sistema embebido.
		\end{itemize}
		\subsubsection{Plataformas de Sistemas Embebidos}
		\begin{itemize}
			\item Identificar las plataformas de sistemas embebidos.
		\end{itemize}
		\subsubsection{Proceso de Selección de Plataformas}
		\begin{itemize}
			\item Explicar el proceso de selección de plataformas de sistemas embebidos de acuerdo a la función del proyecto.... Por que eliguieron una pagina web a una app movil.
		\end{itemize}
		\subsubsection{Proceso de Diseño y Construcción}
		\begin{itemize}
			\item Explicar el proceso de diseño de sistemas embebidos.
			\item Explicar el proceso de construcción de sistemas embebidos.
		\end{itemize}
		\subsubsection{Aplicaciones}
		\begin{itemize}
			\item Explicar las aplicaciones de microcontroladores en la robótica, domótica e internet de las cosas y como se integra en el proyecto... 
		\end{itemize}
	\end{itemize}
	
	
	\subsection{Descripción del proyecto a desarrollar}
	Automatizar procesos que, si se realizan de la manera antigua, resultan bastante tardados, es fundamental para mejorar la eficiencia operativa. Lo que se tiene pensado es aumentar la productividad y disminuir el tiempo estimado para medir parámetros críticos como el pH y la temperatura del agua en los estanques de cultivo. Actualmente, estas mediciones se llevan a cabo de forma manual, lo que representa una pérdida de tiempo y un riesgo para la salud de los peces debido a la falta de registros continuos y alertas tempranas ante desviaciones.
	
	Por esta razón, se propone el desarrollo HYDRAWARE, un sistema automatizado de monitoreo mediante el uso de sensores conectados a un microcontrolador y una aplicación móvil. Este sistema permitirá registrar, visualizar en tiempo real y generar alertas cuando los niveles de los parámetros se encuentren fuera de los límites aceptables, ya sea por valores bajos o elevados. Con esta solución tecnológica se busca mejorar la toma de decisiones, reducir errores humanos, optimizar el manejo del agua y promover una acuicultura más eficiente, precisa y sustentable.
	
	\subsection{Materiales}
	\begin{itemize}
		\item Sensor de Temperatura Digital Ds18b20.
		\item Adaptador para sensor Ds18b20
		\item PH-4502C
		\item Electrodo E201-BNC Tipo de sonda
		\item Display LCD 2x16
	\end{itemize}
	
	\subsection{Descripción de Arduino (ESP32-WROOM-32)}
	El ESP32-WROOM-32 es un módulo de microcontrolador (MCU) con conectividad Wi-Fi, Bluetooth y Bluetooth LE. Está diseñado para una variedad de aplicaciones, desde redes de sensores de baja potencia hasta tareas más exigentes como la codificación de voz y la transmisión de música. El núcleo del módulo es el chip ESP32-D0WDQ6.
	
	\subsubsection{Características principales:}
	\begin{itemize}
		\item \textbf{Conectividad Inalámbrica:}
		\begin{itemize}
			\item Wi-Fi: 802.11 b/g/n, con un rango de frecuencia central de 2412 a 2484 MHz.
			
			\item Bluetooth: Especificación v4.2 BR/EDR y Bluetooth LE. El receptor NZIF tiene una sensibilidad de -97 dBm, y el transmisor es Clase-1, Clase-2 y Clase-3.
		\end{itemize}
		
		\item \textbf{Interfaces del Módulo:}
		\begin{itemize}
			\item Soporta SD card, UART, SPI, SDIO, I2C, LED PWM, Motor PWM, I2S, IR, DAC, TWAI (compatible con CAN 2.0), sensor táctil capacitivo, ADC y contador de pulsos GPIO.
		\end{itemize}
		
		\item \textbf{Hardware Integrado:}
		\begin{itemize}
			\item Cristal de 40 MHz y Flash SPI de 4 MB.
		\end{itemize}
		
		\item \textbf{Condiciones de Operación:}
		\begin{itemize}
			\item Voltaje de operación: 3.0V 3.6V.
			\item Rango de temperatura: -40C +85C.
			\item Corriente de operación promedio: 80 mA.
		\end{itemize}
		\item \textbf{CPU y Memoria Interna:}
		\begin{itemize}
			\item Dos microprocesadores Xtensa® 32-bit LX6 de baja potencia.
			
			\item 520 KB de SRAM en chip, 448 KB de ROM para arranque y 16 KB de SRAM en RTC.
			
			
		\end{itemize}
	\end{itemize}
	
	\subsubsection{Elementos principales:}
	\begin{itemize}
		\item Módulo ESP32-WROOM-32.
		
		\item Chip ESP32-D0WDQ6.
		
		\item Conectividad Wi-Fi y Bluetooth/Bluetooth LE.
		
		\item Flash SPI de 4 MB y cristal oscilador de 40 MHz.
		
		\item Pines GPIO, con los GPIO6 a GPIO11 conectados a la flash SPI y no recomendados para otros usos.
		
		
	\end{itemize}
	
	\subsection{Software utilizado para la interfaz Usuario - Arduino}
	
	\subsection{Justificacion del Software Seleccionado}
	\subsection{Diagrama}
	\subsection{kit de arduino}



\end{document}